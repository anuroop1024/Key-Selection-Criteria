\documentclass{article}
\usepackage[utf8]{inputenc}
\usepackage[a4paper, total={6.5in, 10.5in}]{geometry}
\usepackage{mathptmx}
\usepackage{setspace}


\title{Lecturer - Electrical & Bio-medical Engineering \\Job\# 571985}
\author{Dr. Anuroop Gaddam}


\begin{document}
\onehalfspacing
\maketitle

\section{\textit{\fontsize{12}{12}\selectfont Demonstrated extensive expertise in the field of bio-medical electronics.} }

\fontsize{12}{12}\selectfont As outlined on page 2 of my curriculum vitæ, I have authored 3 articles in reputed journals, 3 book chapters and 18 conference papers with a good citation record. 

\section{\textit{\fontsize{12}{12}\selectfont Demonstrated ability to develop, prepare and deliver high quality curriculum and course materials at tertiary level for undergraduate and post-graduate programs.}}

\fontsize{12}{12}\selectfont My university studies and academic employment experience have involved working extensively in developing high-quality learning resources. Large class with a high number of students present a special challenge for experiential learning. By effectively utilising the online methods to enhance learning skills in these circumstances enable me to achieve positive outcomes. To improve the learning experience for the students bot the under-gradate and postgraduate programs I encouraged a more interactive and self-directed style of learning to develop a range of flexible learning tools. Such methods included recent research, practice examinations and many exercises and their solutions. There were also few simulations of gaming activities where students could participate, obtaining their own results and then comparing with the theoretical outcome. To facilitate student interaction in the context of large numbers of students I created an electronic discussion board where students could exchange their ideas both with faculty and each other. My previous employment as a lecturer at Waikato Institute of Technology has involved working cooperatively in a team of nine people. This position has involved supporting faculty members to current teaching resources, lesson plans, learning outcomes, effective blended delivery methods, implementing flipped classroom techniques and effective assessment methods etc, for both electrical, electronics modules. 



\section{\textit{\fontsize{12}{12}\selectfont Demonstrated ability in leading the development of innovative approaches to course delivery and student-centred learning which successfully exploits new technologies, with a commitment to continuous quality improvement through ongoing training and development.}}

My aim is to develop and introduce creative learning resources to my student learning. This enables students to show their abilities in using electronics, computer programming systems and learn about them in integrated ways.  By doing so will allow them to become enabled to work, study and think creatively. I tend to personalise the learning for groups of students in order to allow them to have adequate access to the associated technologies. This practical hands-on approach will encourage students to become industry ready.  By utilising the latest skills and current technologies in order to develop students proficiencies in latest technologies to the extent that they will be in demand in the current workplace.  And by utilising the e-learning system Moodle interactively, I am able to transcend old boundaries and promote learning in a far more integrated manner. In doing so, my students are able to work both individually and collaboratively as this technology has allowed their tasks to be pursued extensively and in much more depth. This pedagogy enables them to develop and move more quickly towards becoming an independent and contributing member in teams.


\section{\textit{\fontsize{12}{12}\selectfont Excellent interpersonal and communication skills appropriate for interacting with students, staff and industry, together with a strong commitment to teamwork and multidisciplinary collaboration.}}

As a researcher and lecturer, I always demonstrated high-level communication skills as it is necessary to execute my responsibilities effectively. For example, during my early years as a researcher, I won multiple awards for my project presentations \& talks. This involved presenting my research work at international conferences, as outlined on page 4 of my curriculum vitæ. As a lecturer in my previous role, I mentor and confer with students from diverse nationalities and that enabled me to gain high-level verbal communications skills over the years.  In this position, I have also supported team members in developing teaching resources, lesson plans, learning outcomes, effective blended delivery methods and effective assessment methods etc, for electrical, electronics modules. Currently, I am involved in two inter-discipline projects that involve teams from the school of engineering & computer science department, School of architecture and school of education. This involves regular group meetings with colleagues, project team leaders, and industry partners for research outputs, drafting and redrafting the funding applications, research documents, setting goals and creating resources.  

\section{\textit{\fontsize{12}{12}\selectfont Demonstrated capacity to work effectively and sensitively within a culturally diverse environment.}}

Having been working in multicultural work environments made me acquired the essential knowledge in relation to cultural diversity issues, cultural conceptions and perspective of diverse cultures. I recognise culture as a dynamic social phenomenon, and this tends to have an impact on personal behaviour, interpersonal relationships, perception and social expectations of others. This understanding made me recognise the unique way individuals may experience a culture and respond to past experiences and situations. In addition, I am proficient in managing the tasks for applying culturally respectful practices in the workplace and to demonstrate respect and inclusiveness of diverse people in all work practices, form effective workplace relationships with my students and colleagues of diverse backgrounds and cultures. I strive to identify and implement culturally safe work practices, respond respectfully and sensitively to cultural beliefs and practices.

\end{document}
